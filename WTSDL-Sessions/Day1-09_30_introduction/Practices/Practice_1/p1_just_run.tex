\documentclass[a4paper,onecolumn]{article}
\usepackage{lipsum}
\usepackage{amsmath}

%%------Title and Author of the document-----
\title{Article 370 of the Constitution of India}
\author{Type your name}

\begin{document}
\maketitle

\tableofcontents %% This create the table of contents

\section{Introduction}
Article 370 of the Indian constitution gave special status to the region of Jammu and Kashmir, allowing it to have a separate constitution, a state flag and autonomy over the internal administration of the state. It existed until 2019, when it was revoked.

The article was drafted in Part XXI of the Constitution: Temporary, Transitional and Special Provisions. The Constituent Assembly of Jammu and Kashmir, after its establishment, was empowered to recommend the articles of the Indian constitution that should be applied to the state or to abrogate the Article 370 altogether. After consultation with the state's Constituent Assembly, the 1954 Presidential Order was issued, specifying the articles of the Indian constitution that applied to the state. Since the Constituent Assembly dissolved itself without recommending the abrogation of Article 370, the article was deemed to have become a permanent feature of the Indian Constitution.

\section{Purpose}
The state of Jammu \& Kashmir's original accession, like all other princely states, was on three matters: defence, foreign affairs and communications. All the princely states were invited to send representatives to India's Constituent Assembly, which was formulating a constitution for the whole of India. They were also encouraged to set up constituent assemblies for their own states.

	\subsection{This is some subsection}
	 Most states were unable to set up assemblies in time, but a few states did, in particular Saurashtra Union, Travancore-Cochin and Mysore. Even though the States Department developed a model constitution for the states, in May 1949, the rulers and chief ministers of all the states met and agreed that separate constitutions for the states were not necessary. They accepted the Constitution of India as their own constitution. The states that did elect constituent assemblies suggested a few amendments which were accepted. The position of all the states (or unions of states) thus became equivalent to that of regular Indian provinces. In particular, this meant that the subjects available for legislation by the central and state governments was uniform across India
	 
	 \subsection{This is some other subsection}
	 In the case of Jammu and Kashmir, the representatives to the Constituent Assembly requested that only those provisions of the Indian Constitution that corresponded to the original Instrument of Accession should be applied to the State. Accordingly, the Article 370 was incorporated into the Indian Constitution, which stipulated that the other articles of the Constitution that gave powers to the Central Government would be applied to Jammu and Kashmir only with the concurrence of the State's constituent assembly. This was a "temporary provision" in that its applicability was intended to last till the formulation and adoption of the State's constitution. However, the State's constituent assembly dissolved itself on 25 January 1957 without recommending either abrogation or amendment of the Article 370. Thus the Article has become a permanent feature of the Indian constitution, as confirmed by various rulings of the Supreme Court of India and the High Court of Jammu and Kashmir, the latest of which was in April 2018
\section{Conclusion}
On 5 August 2019, the Home Minister Amit Shah introduced a bill in the Rajya Sabha to convert Jammu and Kashmir's status of a state to two separate union territories, namely Jammu and Kashmir, and Ladakh. The union territory of Jammu and Kashmir is proposed to have a legislature under the bill whereas the union territory of Ladakh is proposed to not have one. By the end of the day, the bill was passed by Rajya Sabha with 125 votes in its favour. The next day, the bill was passed by Lok Sabha with 370 votes in its favour.




\end{document}