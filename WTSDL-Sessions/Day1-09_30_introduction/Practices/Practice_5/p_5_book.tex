\documentclass[oneside]{book}
\usepackage{amsmath, amsthm}


\theoremstyle{definition}
\newtheorem{definition}{Definition}[chapter]

\theoremstyle{default}
\newtheorem{theorem}{Theorem}[chapter]




\title{This is the title of the Book}
\author{Your Name}
\date{}

\begin{document}
	\maketitle
	
	
%	\part{This is first part}
	
	\chapter{Introduction}
	 Curabitur ullamcorper ultricies nisi. Nam eget dui. Etiam rhoncus. Maecenas tempus, tellus eget condimentum rhoncus, sem quam semper libero, sit amet adipiscing sem neque sed ipsum. Nam quam nunc, blandit vel, luctus pulvinar, hendrerit id, lorem. Maecenas nec odio et ante tincidunt tempus. Donec vitae sapien ut libero venenatis faucibus. 
	
	\begin{definition}[ faucibus tincidunt]
	Nullam quis ante. Etiam sit amet orci eget eros faucibus tincidunt. Duis leo. Sed fringilla mauris sit amet nibh. Donec sodales sagittis magna.
	\end{definition}
	
	
	\chapter{Literature Review}
	qui dolorem ipsum quia dolor sit amet, consectetur, adipisci velit, sed quia non numquam eius modi tempora incidunt ut labore et dolore magnam aliquam quaerat voluptatem. Ut enim ad minima veniam, quis nostrum exercitationem ullam corporis suscipit laboriosam, nisi ut aliquid ex ea commodi consequatur? Quis autem vel eum iure reprehenderit qui in ea voluptate velit esse quam nihil molestiae consequatur.
		\begin{theorem}
		qui blanditiis praesentium voluptatum deleniti atque corrupti quos dolores et quas molestias excepturi sint occaecati cupiditate non provident, similique sunt in culpa qui officia deserunt mollitia animi, id est laborum et dolorum fuga. Et harum quidem rerum facilis est et expedita distinctio. Nam libero tempore, cum soluta nobis est eligendi optio cumque nihil impedit quo minus id quod maxime placeat facere
		\end{theorem}
	
	
	\chapter{Methodology}
	
	Fill the chapter using the text in the text file names practise5dummyText.txt
	
	
	\chapter{Main Results}
	
	
	
	\chapter{Conclusion}
	
	

\end{document}