\documentclass[oneside,12pt]{article}

\title{Moral Stories Collection}
\author{Author Name}
\begin{document}
	\maketitle
\tableofcontents

\section{A Hole in the Fence}
In a small village, a little boy lived with his father and mother. He was the only son.The parents of the little boy were very depressed due to his bad temper. The boy used to get angry very soon and taunt others with his words. His bad temper made him use words that hurt others. He scolded kids, neighbours and even his friends due to anger. His friends and neighbours avoided him, and his parents were really worried about him.

His mother and father advised him many times to control his anger and develop kindness. Unfortunately, all their attempts failed. Finally, the boy’s father came up with an idea.

One day, his father gave him a huge bag of nails. He asked his son to hammer one nail to the fence every time he became angry and lost his temper. The little boy found it amusing and accepted the task.

Every time he lost his temper, he ran to the fence and hammered a nail. His anger drove him to hammer nails on the fence 30 times on the first day! After the next few days, the number of nails hammered on the fence was reduced to half. The little boy found it very difficult to hammer the nails and decided to control his temper.

Gradually, the number of nails hammered to the fence was reduced and the day arrived when no nail was hammered! The boy did not lose his temper at all that day. For the next several days, he did not lose his temper, and so did not hammer any nail.

Now, his father told him to remove the nails each time the boy controlled his anger. Several days passed and the boy was able to pull out most of the nails from the fence. However, there remained a few nails that he could not pull out.

The boy told his father about it. The father appreciated him and asked him pointing to a hole, “What do you see there?”

The boy replied, “a hole in the fence!”

He told the boy, “The nails were your bad temper and they were hammered on people. You can remove the nails but the holes in the fence will remain. The fence will never look the same. It has scars all over. Some nails cannot even be pulled out. You can stab a man with a knife, and say sorry later, but the wound will remain there forever. Your bad temper and angry words were like that! Words are more painful than physical abuse! Use words for good purposes. Use them to grow relationships. Use them to show the love and kindness in your heart!”

\begin{quote}
\textbf{Moral} – Unkind words cause lasting damage: Let our words be kind and sweet.
\end{quote}

\section{Change Yourself and not The World}
Long ago, people lived happily under the rule of a king. The people of the kingdom were very happy as they led a very prosperous life with an abundance of wealth and no misfortunes.

Once, the king decided to go visiting places of historical importance and pilgrim centres at distant places. He decided to travel by foot to interact with his people. People of distant places were very happy to have a conversation with their king. They were proud that their king had a kind heart.

After several weeks of travel, the king returned to the palace. He was quite happy that he had visited many pilgrim centres and witnessed his people leading a prosperous life. However, he had one regret.

He had intolerable pain in his feet as it was his first trip by foot covering a long distance. He complained to his ministers that the roads weren’t comfortable and that they were very stony. He could not tolerate the pain. He said that he was very much worried about the people who had to walk along those roads as it would be painful for them too!

Considering all this, he ordered his servants to cover the roads in the whole country with leather so that the people of his kingdom can walk comfortably.

The king’s ministers were stunned to hear his order as it would mean that thousands of cows would have to be slaughtered in order to get sufficient quantity of leather. And it would cost a huge amount of money also.

Finally, a wise man from the ministry came to the king and said that he had another idea. The king asked what the alternative was. The minister said, “Instead of covering the roads with leather, why don’t you just have a piece of leather cut in appropriate shape to cover your feet?”

The king was very much surprised by his suggestion and applauded the wisdom of the minister. He ordered a pair of leather shoes for himself and requested all his countrymen also to wear shoes.

\begin{quote}
\textbf{Moral:} Instead of trying to change the world, we should try to change ourselves.
\end{quote}

%%%%%-- Can you add the following story also?? ----

%Strong or Weak
%
%There was a proud teak tree in the forest. He was tall and strong. There was a small herb next to the tree.
%
%The teak tree said, “I am very handsome and strong. No one can defeat me.” Hearing this, the herb replied, “Dear friend, too much pride is harmful. Even the strong will fall one day.”
%
%The teak ignored the herb’s words. He continued to praise himself.
%
%A strong wind blew. The teak stood firmly. Even when it rained, the teak stood strong by spreading its leaves.
%
%During these times, the herb bowed low. The teak made fun of the herb.
%
%One day, there was a storm in the forest. The herb bowed low. As usual, the teak did not want to bow.
%
%The storm kept growing stronger. The teak could no longer bear it. He felt his strength giving way.
%
%He tried his best to stand upright, but in the end, he fell down. That was the end of the proud tree.
%
%When everything was calm again, the herb stood straight. He looked around. He saw that the proud teak had fallen.
%
%Moral: Pride goes before a fall.

\end{document}