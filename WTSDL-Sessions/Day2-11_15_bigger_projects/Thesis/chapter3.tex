\chapter{2-Path  Signed Graphs}
\section{2-Path Product  Signed Graphs}
Let $G=(V,E,\sigma)$ be a signed graph.
The 2-path product signed graph $G\hat{\#}G= (V, E^\prime,\sigma^\prime )$ is defined as
follows: The vertex set is same as the original signed graph
$G$ and two vertices $u,v\in V(G\hat{\#}G)$, are adjacent if and only
if there exists a $uv$-path of length two in $G$. The sign $\sigma^\prime(uv)=\mu(u)\mu(v)$, $\mu$ is the canonical marking.
The Figure \ref{fig3.2} is the 2-path product signed graph of the signed graph $K^-_{4,1}$ in the Figure \ref{fig3.1}.
% \begin{minipage}{200pt}
\begin{figure}[h!]
\begin{pspicture}(4,4)
\rput(4.5,-.75){\scalebox{1}{
\pscircle[linewidth=.05, linecolor=blue](1,1){.25}
\pscircle[linewidth=.05, linecolor=blue](1,4){.25}
\pscircle[linewidth=.05, linecolor=blue](4,1){.25}
\pscircle[linewidth=.05, linecolor=blue](4,4){.25}
\pscircle[linewidth=.05, linecolor=blue](2.5,2.5){.25}

\psline[linewidth=.05](1.25,1)(2.25,2.5)
\psline[linewidth=.05](1.25,4)(2.25,2.5)
\psline[linewidth=.05](3.75,1)(2.75,2.5)
\psline[linewidth=.05](3.75,4)(2.75,2.5)

\rput(.95,1){ $4$}
\rput(.95,4){ $5$}
\rput(3.95,4){ $2$}
\rput(3.95,1){ $3$}
\rput(2.45,2.5){ $1$}

\rput(3.6,3){$-1$}
\rput(1.4,3){$-1$}
\rput(1.4,1.75){$-1$}
\rput(3.6,1.75){$-1$}
}}
\end{pspicture}
\caption{\label{fig3.1} The signed graph $K^-_{4,1}$.}
\end{figure}

\begin{figure}
\begin{pspicture}(4,4)
\rput(4,0){\scalebox{1}{
\pscircle[linewidth=.05, linecolor=blue](1,1){.25}
\pscircle[linewidth=.05, linecolor=blue](1,4){.25}
\pscircle[linewidth=.05, linecolor=blue](4,1){.25}
\pscircle[linewidth=.05, linecolor=blue](4,4){.25}
\pscircle[linewidth=.05, linecolor=blue](6.5,2.5){.25}

\psline[linewidth=.05](1.25,1)(3.75,1)
\psline[linewidth=.05](1.25,4)(3.75,4)
\psline[linewidth=.05](1,1.25)(1,3.75)
\psline[linewidth=.05](4,1.25)(4,3.75)
\psline[linewidth=.05](1.15,1.15)(3.85,3.85)
\psline[linewidth=.05](1.15,3.85)(3.85,1.15)
\rput(.95,1){ $4$}
\rput(.95,4){ $5$}
\rput(3.95,4){ $2$}
\rput(3.95,1){ $3$}
\rput(6.45,2.5){ $1$}

\rput(.75,2.5){$+1$}
\rput(4.25,2.5){$+1$}
\rput(2.5,4.25){$+1$}
\rput(2.5,.75){$+1$}
\rput(2.25,3.20){$+1$}
\rput(3.25,2.9){$+1$}
}}
\end{pspicture}
\caption{\label{fig3.2} The  2-path product signed graph of $K^-_{4,1}$.}
\end{figure}
The following result  about  2-path product signed graph is cited from \cite{SIN17}.
\begin{proposition}
2-path product signed graph of a signed graph
S is always balanced.
\end{proposition}
We noticed the following about canonical consistency of  the 2-path product signed graph of a signed graph.
\begin{proposition}
2-path product signed graph of a signed graph
S is always canonically consistent.
\end{proposition}

\section{2-Path  Signed Graphs}
Let $G=(V,E,\sigma)$ be a  signed graph.
The {\it $n$-path  signed graph} $G{\#}G= (V, E^\prime,\sigma^\prime )$ is defined as
follows: The vertex set is same as the original signed graph
$G$ and two vertices $u,v\in V(G{\#}G)$, are adjacent if and only
if there exists a $uv$-path of length $n$ in $G$. The sign $\sigma^\prime(uv)$ is $-1$ whenever in every $uv$-path of length $n$ 
in $G$ all edges are negative.

Let $G$ be any signed graph and $G\# G$ be the 2-path signed graph of $G$.
Define 
\begin{eqnarray*}
S&=&\{G~|~G \text{ is balanced and canonically consistent}\}\\
S_1&=&\{G\in S~|~G\# G \text{ is balanced and canonically consistent}\}\\
S_2&=&\{G\in S~|~G\# G \text{ is balanced but not canonically consistent}\}\\
S_3&=&\{G\in S~|~G\# G \text{ is canonically consistent but not balanced}\}\\
S_4&=&\{G\in S~|~G\# G \text{ is neither canonically consistent nor balanced}\} 
\end{eqnarray*}
The following proposition shows that $S_1\neq\Phi$.
\begin{proposition}
 For any graph $G$, $G^+\# G^+$ is balanced and canonically consistent.
\end{proposition}
\begin{proof}
 For any graph $G$, $G^+$ denotes the corresponding signed graph with all edges positive and so $G\in S$. So,  all edges in $G^+\# G^+$
 are positive. Hence the result.
\end{proof}
% \begin{proposition}
%   If $G$ has no adjacent negative edges then $G\in S_1$.
% \end{proposition}
% \begin{proof}
%  Let $G$ be a graph which has no adjacent negative edges. Then $G\#G$ will not have any negative edge because the edge in 
% $G\#G$ is negative iff all the edges in all two-path in $G$ are negative. Hence, all the edges are positive so $G\#G$  is $B$ and 
% $CC$. Therefore $G\in S_1$.
% \end{proof}
% % 
% \begin{proposition}
%  If $G$ have adjacent negative edges${u,v},{v,w}$ where $u,v,w\in V(G)$ but $\exists$ another 2-path between $u$ and $w$ where 
% both 
% edges are not negative then $G\in S_1$.
% \end{proposition}
% \begin{proof}
%  Let $G$ be a graph. By the definition of $G\#G$ the edge in $G\#G$ is negative iff all the edges in all two-path in $G$ are 
% negative. Since there exists another 2-path for which both edges are not negative then $G\sharp G$ will have all edge positive. 
% So $G\sharp G$ 
% is $B$ and $CC$. Hence $G\in S_1$.
% \end{proof}
\remark $C^-_{2n+1}\not\in S$ but $C^-_{2n}\in S$.

\begin{proposition}
$C^-_{2n}\in S_1$ if and only if $n$ is even.
\end{proposition}
\begin{proof}
First suppose that $C^-_{2n}\in S_1$. We need to show that $n$ is even. 
If possible let $n$ be odd. Then $C^-_{2n}\# C^-_{2n}$ will consist of two copies of $C^-_n$. Since $n$ is odd, so the component $C^-_{n}$ 
will have sign $-1$.
Hence $C^-_{2n}\# C^-_{2n}$ is not balanced, a contradiction. So, $n$ must be even.

Conversely, suppose that $n$ is even. We consider the following cases:

CASE 1: $n= 2$. 

In this case $C^-_{2n}\# C^-_{2n}$ is acyclic and so it must be balanced and canonically consistent.

% CASE 2:
% For $n= 2$ we get cycle $C^-_4$
% Figure(3.1*************)
% \begin{figure}[h!]
%  \begin{pspicture}(7,6)
% \rput(4,0){\scalebox{1}{
% \pscircle[linewidth=.05, linecolor=blue](1,1){.25}
% \pscircle[linewidth=.05, linecolor=blue](1,5){.25}
% \pscircle[linewidth=.05, linecolor=blue](5,1){.25}
% \pscircle[linewidth=.05, linecolor=blue](5,5){.25}
% 
% 
% \psline[linewidth=.05](1.25,1)(4.75,1)
% \psline[linewidth=.05](1.25,5)(4.75,5)
% \psline[linewidth=.05](1,4.75)(1,1.25)
% \psline[linewidth=.05](5,4.75)(5,1.25)
% }
% \rput(.95,1){ $4$}
% \rput(.95,5){ $1$}
% \rput(4.95,5){ $2$}
% \rput(4.95,1){ $3$}
% }
% \end{pspicture}
% \caption{\label{fig2.0} .}
% \end{figure}
% 
% Its 2-path signed graph $C^-_4\sharp C^-_4$ is
% Figure(3.2) .
% \begin{figure}[h!]
%  \begin{pspicture}(7,6)
% \rput(4,0){\scalebox{1}{
% \pscircle[linewidth=.05, linecolor=blue](1,1){.25}
% \pscircle[linewidth=.05, linecolor=blue](1,5){.25}
% \pscircle[linewidth=.05, linecolor=blue](5,1){.25}
% \pscircle[linewidth=.05, linecolor=blue](5,5){.25}
% 
% 
% \psline[linewidth=.05](1.2,1.1)(4.9,4.8)
% \psline[linewidth=.05](1.2,4.9)(4.9,1.2)
% }
% \rput(.95,1){ $4$}
% \rput(.95,5){ $1$}
% \rput(4.95,5){ $2$}
% \rput(4.95,1){ $3$}
% }
% \end{pspicture}
% \caption{\label{fig2.0}.}
% \end{figure}
% Since there is no cycle so it is $B$ and $CC$. Hence $C^-_4\in S_1$.


CASE 2: $n\geq 4$ i.e. $n= 4,6,8,\ldots$
Here for each $n$, $C^-_{2n}\sharp C^-_{2n}$ consist of two copies of $C^-_n$. Since for even $n$, $C^-_n$ is balanced and canonically consistent,
so $C^-_{2n}\# C^-_{2n}$ is balanced and canonically consistent. Hence $C^-_{2n}\in S_1$.
\end{proof}
\remark For any positive integer $n$, let $\Sigma=(K_{n,1},\sigma)$. Then $\Sigma\# \Sigma$ will consist of two components with 
underlying graphs
 as $K_{n}$ and $K_1$. Further, the component $K_1$ will consist of the center of $K_{n,1}$.

\remark For $n=1,2$, if $\Sigma=(K_{n,1},\sigma)$ then $\Sigma\# \Sigma$ will be acyclic and so $\Sigma\in S_1$.
\begin{proposition}
  If $\Sigma=(K_{n,1},\sigma)$ and $n\geq 3$, then $\Sigma\in S_1$ if and only if $\Sigma$ has at most one edge with negative sign.
  \end{proposition}
  \begin{proof}
   First suppose that, $\Sigma$ has at most one edge with negative sign. Then each edge of $\Sigma\# \Sigma$ will be positive 
   and the result follows.
   
   Conversely, suppose $\Sigma\in S_1$. If possible let, $\Sigma$ has more than 
one negative edge.
   Let $v$ be the center and $u_1,u_2,u_3$ be any three pendent vertices.
   Then the vertices $u_1,u_2,u_3$ will form a triangle in $\Sigma\# \Sigma$.
   So, it is enough to consider the following cases:
   
   CASE 1: $\Sigma$ has two negative edges. In particular, let the edges $\{v,u_1\}$ and $\{v,u_2\}$ be negative. 
   In this case, the triangle $u_1u_2u_3u_1$ in $\Sigma\# \Sigma$ will have one negative edge $\{u_1,u_2\}$ and so  
$\Sigma\# \Sigma$ is not balanced.
   
   CASE 2: $\Sigma$ has three negative edges. In particular, let the edges 
$\{v,u_1\}, \{v,u_2\}$ and $\{v,u_3\}$ be negative.  In this case, the triangle 
$u_1u_2u_3u_1$ in $\Sigma\# \Sigma$ will have all edges negative 
 and so  $\Sigma\# \Sigma$ is not balanced.

In both the cases, $\Sigma\# \Sigma$ not balanced, a contradiction. So, 
$\Sigma$ can have at most one edge with negative sign.   
  \end{proof}



% \begin{proof}
% %  We would prove it by method of Mathematical Induction. Let $G$ be a graph
% %  
% % Base step: for $n= 2$
% % Star graph $G$ will be Fig(3.3).
% % \begin{figure}[h!]
% %  \begin{pspicture}(3,3)
% % \rput(4,0){\scalebox{1}{
% % \pscircle[linewidth=.05, linecolor=blue](1,1){.25}
% % \pscircle[linewidth=.05, linecolor=blue](5,1){.25}
% % 
% % 
% % \psline[linewidth=.05](1.25,1)(4.75,1)
% % }
% % \rput(4.95,1){ $2$}
% % \rput(.95,1){ $1$}
% % }
% % \end{pspicture}
% % \caption{\label{fig2.0} A star Graph with two vertices .}
% % \end{figure}
% % 2-path signed graph $G\sharp G$ will be Fig(3.4).
% % 
% % \begin{figure}[h!]
% %  \begin{pspicture}(3,3)
% % \rput(4,0){\scalebox{1}{
% % \pscircle[linewidth=.05, linecolor=blue](1,1){.25}
% % \pscircle[linewidth=.05, linecolor=blue](5,1){.25}
% % }
% % \rput(4.95,1){ $2$}
% % \rput(.95,1){ $1$}
% % }
% % \end{pspicture}
% % \caption{\label{fig2.0} 2- path graph of star Graph with two vertices .}
% % \end{figure}
% % It is $K_1\cup K_1$.
% % 
% % For $n= 3$.Star graph $G$ will be Fig(3.5)
% % \begin{figure}[h!]
% %  \begin{pspicture}(3,3)
% % \rput(4,0){\scalebox{1}{
% % \pscircle[linewidth=.05, linecolor=blue](1,1){.25}
% % \pscircle[linewidth=.05, linecolor=blue](5,1){.25}
% % \pscircle[linewidth=.05, linecolor=blue](3,1){.25}
% % 
% % 
% % \psline[linewidth=.05](1.25,1)(2.75,1)
% % \psline[linewidth=.05](3.25,1)(4.75,1)
% % }
% % \rput(4.95,1){ $3$}
% % \rput(.95,1){ $2$}
% % \rput(2.95,1){ $1$}
% % }
% % \end{pspicture}
% % \caption{\label{fig2.0} A star Graph with 3-vertices .}
% % \end{figure}
% % 2-path signed graph $G\sharp G$ will be Fig(3.6)
% % 
% % \begin{figure}[h!]
% %  \begin{pspicture}(3,3)
% % \rput(4,0){\scalebox{1}{
% % \pscircle[linewidth=.05, linecolor=blue](1,1){.25}
% % \pscircle[linewidth=.05, linecolor=blue](5,1){.25}
% % \pscircle[linewidth=.05, linecolor=blue](3,1){.25}
% % 
% % 
% % \psline[linewidth=.05](1.25,1)(2.75,1)
% % }
% % \rput(4.95,1){ $1$}
% % \rput(.95,1){ $2$}
% % \rput(2.95,1){ $3$}
% % }
% % \end{pspicture}
% % \caption{\label{fig2.0} A 2-path signed graph of star Graph with 3-vertices .}
% % \end{figure}
% % 
% % It is $K_2\cup K_1$.
% % Hence it is true for $n= 2,3 $.
% % 
% % Induction step: Let us consider it is true for $n$ - vertices. That is for $n$-vetices 2-path signed graph is $K_{n-1}\cup K_1$.
% % Now we will check for $n+1$ vertices. Let us consider a star graph with $n+1$ vertices.
% % Since 2-path signed graph of star graph with $n$-vertices has $K_{n-1}\cup K_1$so when we add one 
% % vertex to graph which has edge with $v_1$ center vertex. Since $v_1$ is connected to each of the vertices $v_1,v_2,\ldots,v_n+1$ hence 
% % $\exists$ a 2-path between all vertices except $v_1$.
% % Hence 2-path signed graph of $n+1$ vertex star graph will have $n$ complete with $v_1$ isolated i.e. $K_n\cup K_1$.
% % 
% % Hence it is true for all $n$.
% % 
% % Hence proved.
% \end{proof}

\remark 2-path signed graph of $C_{2n+1}$ is isomorphic to itself.
\begin{theorem}
If there are even number of consecutive negative edges in a cycle other than \{ $C_{2n}^-,~ n$ is even\} and ${C_4}$ then it will 
belong to $S_3$.
\end{theorem}
\begin{proof}
 Let us consider there is even number of consecutive negative edges in a cycle $C_n$, where $n=3,4,5,\ldots$ \}
 
CASE 1: If $n$ is odd.
Then 2-path signed graph of $C_n$ is isomorphic to itself.
For even number of consecutive negative edges $2k$ (say) its 2-path signed graph will have $2k-1$ negative edges, i.e. odd number 
of negative edges in cycle $C_n$ i.e.$C_n\# C_n$ is not balanced but canonically consistent. 
Hence, $C_n\in S_3$

CASE 2: If $n$ is even.

For $n=4$ in any case 2-path signed graph of $C_4$ does not contains any cycle hence it is balanced and canonically consistent.

Again we know that a graph $C^-_{2n}\in S_1$ if and only if n is even.

Now for other cycles $C_n$, its 2-path signed graph will consists two copies of $C_{n/2}$ .
If component $C_{n/2}$ is odd cycle then same as CASE 1 we can say that in either of component $C_{n/2}$ there is odd number of 
negative edges, i.e. $C_n\# C_n$  is not balanced but canonically consistent.
Hence, $C_n\in S_3$

If $C_{n/2}$ is even cycle, then it will be less than n. Now with same argument as above either of component $C_{n/2}$ will have 
odd number of negative edges i.e. $C_n\# C_n$  is not balanced but canonically consistent.
Hence, $C_n\in S_3$.
\end{proof}

\begin{proposition}
$C^-_{2n}\in S_3$ if and only if $n$ is odd.
\end{proposition}
\begin{proof}
First suppose that $C_{2n}^-\in S_3$. We need to show that n is odd. If possible let n be even. Then $C_{2n}^-\# C_{2n}^-$  will 
consist of two copies of $C_n^-$. Since $n$ is even so the component $C_n^-$ will be balanced, a contradiction. So $n$ must be 
odd.

Conversely, suppose that $n$ is odd. Then $C_{2n}^-\# C_{2n}^-$  will consist of a pair of disjoint cycles $C_n^-$. Since n is 
odd $C_n^-$ is not balanced but canonically consistent. So $C_{2n}^-\# C_{2n}^-$   is not balanced but canonically consistent. 
Hence          $C_{2n}^-\in S_3$. 
\end{proof}

\remark If $\Sigma=(C_n,\sigma)$, then $n\geq 3$ the underlying graph of $\Sigma\#\Sigma$ will consist of either a cycle or a pair 
disjoint cycle. So, $\Sigma\#\Sigma$ will be canonically consistent and hence $\Sigma\not\in S_4$.

\begin{figure}[h!]
\begin{pspicture}(7,6)
\rput(4,0){\scalebox{1}{
\pscircle[linewidth=.05](1,1){.25}
\pscircle[linewidth=.05](1,5){.25}
\pscircle[linewidth=.05](5,1){.25}
\pscircle[linewidth=.05](5,5){.25}

\psline[linewidth=.05](1,4.75)(1,1.25)
\psline[linewidth=.05](1.25,1.25)(4.75,4.75)
\psline[linewidth=.05](1.25,5)(4.75,5)
\psline[linewidth=.05](5,4.75)(5,1.25)

}
\rput(.95,1){ $4$}
\rput(.95,5){ $1$}
\rput(4.95,5){ $2$}
\rput(4.95,1){ $3$}
\rput(.5,3){ $-1$}
\rput(3,5.5){ $+1$}
\rput(3.5,3.3){ $-1$}
\rput(5.3,3){ $+1$}
}
\end{pspicture}
\caption{\label{fig3.3} The signed graph $\Sigma_1$.}
\end{figure}
\rm
The 2-path signed graph of the signed graph $\Sigma_1$ in the Figure \ref{fig3.3} is shown in the Figure
\ref{fig3.4} and it is evident that $\Sigma_1\in S_4$. Hence $S_4\neq \Phi$.

\begin{figure}[h!]
\begin{pspicture}(7,6)
\rput(4,0){\scalebox{1}{
\pscircle[linewidth=.05](1,1){.25}
\pscircle[linewidth=.05](1,5){.25}
\pscircle[linewidth=.05](5,1){.25}
\pscircle[linewidth=.05](5,5){.25}

\psline[linewidth=.05](1,4.75)(1,1.25)
\psline[linewidth=.05](1.15,1.15)(4.85,4.85)
\psline[linewidth=.05](1.25,5)(4.75,5)
\psline[linewidth=.05](1.15,4.85)(4.85,1.15)
\psline[linewidth=.05](1.25,1)(4.75,1)

}
\rput(.95,1){ $4$}
\rput(.95,5){ $1$}
\rput(4.95,5){ $2$}
\rput(4.95,1){ $3$}
\rput(.5,3){ $+1$}
\rput(3,5.5){ $-1$}
\rput(3.8,3.3){$+1$}
\rput(2.4,4){$+1$}
\rput(3,.5){$+1$}
}
\end{pspicture}
\caption{\label{fig3.4} The 2-path signed graph of $\Sigma_1$.}
\end{figure}
% \section{Consistent Signed Graphs
% The  mark $\mu(\Sigma^\prime)$ of  a  nonempty  
% subgraph  $\Sigma^\prime$
% of  $\Sigma_\mu$  is then defined as the product of the marks of the vertices  in $\Sigma^\prime$.  A  cycle    
% $Z$    in  $\Sigma_\mu$ is  said  to  be    consistent  if    $\mu(Z)=+1$;  otherwise,  it  is  said  to  be  
% inconsistent. 
% Further, $\Sigma_\mu$   is  said  to  be  consistent if every cycle in it is consistent. Otherwise it is said to be inconsistent.
% If product of the sign of the vertices in every cycle of a marked graph is positive then the 
% marked graph is said to be
% consistent.
% \begin{figure}[h!]
% \begin{pspicture}(8,4)
% \rput(2,0){\scalebox{1}{
% \pscircle[linewidth=.05, linecolor=blue](1,1){.25}
% \pscircle[linewidth=.05, linecolor=blue](3,1){.25}
% \pscircle[linewidth=.05, linecolor=blue](5,1){.25}
% \pscircle[linewidth=.05, linecolor=blue](7,1){.25}
% \pscircle[linewidth=.05, linecolor=blue](3,3){.25}
% 
% \psline[linewidth=.05](1.3,1)(2.7,1)
% \psline[linewidth=.05](3.3,1)(4.7,1)
% \psline[linewidth=.05](5.3,1)(6.7,1)
% \psline[linewidth=.05](3,1.3)(3,2.7)
% \psline[linewidth=.05](3.2,2.8)(6.8,1.2)
% \rput(.95,1){ $1$}
% \rput(3,1){ $2$}
% \rput(5,1){ $3$}
% \rput(7,1){ $4$}
% \rput(3,3){ $5$}
% \linenumbers
% \rput(2,.7){ $+1$}
% \rput(4,.7){ $-1$}
% \rput(6,.7){ $-1$}
% \rput(3.3,2){ $+1$}
% \rput(4.9,2.3){ $+1$}
% \rput(.3,1){+1}
% \rput(3,.5){-1}
% \rput(5,.5){+1}
% \rput(7.5,1){-1}
% \rput(3,3.5){+1}}}
% \end{pspicture}\linenumbers
% \caption{\label{fig3.2} }
% \end{figure}
% \section{Canonically Consistent Signed Graphs}
% Every signed graph $\sum=(G,\sigma)$ is associated with a natural 
% marking $$\mu~:~ V(G)\rightarrow\{-1,1\}$$ define as follows:                                                                  
%                                       $$\text {for any vertex } v\in V(G),~\mu (v)=\prod\limits_{u\in N(v)}\sigma(uv).$$ 
%                                       The marking $\mu$ is known as the canonical marking of the signed graph $\sum$. 
% If the signed graph $\sum=(G,\sigma)$ is consistent with respect to the canonical 
% marking then it is called a {\it canonically consistent signed graph.} 
% \begin{figure}[h!]
% \begin{pspicture}(8,4)
% \rput(2,0){\scalebox{1}{
% \pscircle[linewidth=.05, linecolor=blue](1,1){.25}
% \pscircle[linewidth=.05, linecolor=blue](3,1){.25}
% \pscircle[linewidth=.05, linecolor=blue](5,1){.25}
% \pscircle[linewidth=.05, linecolor=blue](7,1){.25}
% \pscircle[linewidth=.05, linecolor=blue](3,3){.25}
% 
% 
% \psline[linewidth=.05](1.3,1)(2.7,1)
% \psline[linewidth=.05](3.3,1)(4.7,1)
% \psline[linewidth=.05](5.3,1)(6.7,1)
% \psline[linewidth=.05](3,1.3)(3,2.7)
% \psline[linewidth=.05](3.2,2.8)(6.8,1.2)
% \rput(.95,1){ $1$}
% \rput(3,1){ $2$}
% \rput(5,1){ $3$}
% \rput(7,1){ $4$}
% \rput(3,3){ $5$}
% 
% \rput(2,.7){ $+1$}
% \rput(4,.7){ $-1$}
% \rput(6,.7){ $-1$}
% \rput(3.3,2){ $+1$}
% \rput(4.9,2.3){ $+1$}
% 
% \rput(.3,1){+1}
% \rput(3,.5){-1}
% \rput(5,.5){+1}
% \rput(7.5,1){-1}
% \rput(3,3.5){+1}
% }}
% \end{pspicture}
% \caption{\label{fig3.1} }
% \end{figure}
% The signed graph in the Figure \ref{fig3.1} represents a 
% canonically consistent signed graph. But the signed graph in the Figure \ref{fig4.1} is not
% canonically consistent. In this chapter, we tried to characterize certain classes of signed graphs that are canonically 
% consistent.
% \begin{figure}[h!]
% \begin{pspicture}(8,4)
% \rput(2,0){\scalebox{1}{
% \pscircle[linewidth=.05, linecolor=blue](1,1){.25}
% \pscircle[linewidth=.05, linecolor=blue](3,1){.25}
% \pscircle[linewidth=.05, linecolor=blue](5,1){.25}
% \pscircle[linewidth=.05, linecolor=blue](7,1){.25}
% \pscircle[linewidth=.05, linecolor=blue](3,3){.25}
% 
% 
% \psline[linewidth=.05](1.3,1)(2.7,1)
% \psline[linewidth=.05](3.3,1)(4.7,1)
% \psline[linewidth=.05](5.3,1)(6.7,1)
% \psline[linewidth=.05](3,1.3)(3,2.7)
% \psline[linewidth=.05](3.2,2.8)(6.8,1.2)
% \rput(.95,1){ $1$}
% \rput(3,1){ $2$}
% \rput(5,1){ $3$}
% \rput(7,1){ $4$}
% \rput(3,3){ $5$}
% 
% \rput(2,.7){ $+1$}
% \rput(4,.7){ $-1$}
% \rput(6,.7){ $-1$}
% \rput(3.3,2){ $+1$}
% \rput(4.9,2.3){ $+1$}
% 
% \rput(.3,1){+1}
% \rput(3,.5){-1}
% \rput(5,.5){+1}
% \rput(7.5,1){-1}
% \rput(3,3.5){+1}
% }}
% \end{pspicture}
% \caption{\label{fig4.1} }
% \end{figure}
% \begin{proposition}\label{prop1}
%  Every signed cycle is canonically consistent.
% \end{proposition}
% \begin{proof}
% Let $\Sigma=(C,\sigma)$ be any signed graph, where $C$ is a cycle. We need to prove that $\Sigma$ is canonically consistent. Let 
% $\Sigma_\mu=(\Sigma, \mu)$ be the canonical marked signed graph of $\Sigma$.
% If $R(\sigma)=\{+\}$ or $R(\sigma)=\{-\}$, then every vertex of $\Sigma$ will get mark $+$ with respect to canonical marking and 
% so it must be canonically consistent.
% 
% 
% So, let $\Sigma$ has both positive and negative edges.
% We apply induction on the number of negative edges to prove this result.
% Let $\Sigma$ has $n$ negative edges.
% \linenumbers
% Base Step: Let $n=1$ and $e=uv$ be the negative edge in $G$. Then according to canonical marking every vertex will get mark $+$ 
% except the vertices $u,v$ in $\Sigma$. Hence the number of vertices with negative marking in $\Sigma_\mu$ is even and so it 
% is consistent. Hence $\Sigma$ is canonically consistent.
% 
% Induction Step: As an induction hypothesis, assume that the result holds for fewer than $n$ negative edges in 
% $\Sigma$ and $e=uv$ be any edge in $\Sigma$  with negative sign. Let $\Sigma^\prime$ be the signed graph obtained from $\Sigma$ by 
% switching the sign of $e$ to $+$. Thus $\Sigma^\prime$ has $n-1$ negative edges and so by induction hypothesis it is canonically 
% consistent. But mark of the vertices in $\Sigma_\mu$ and $\Sigma^\prime_\mu$ differs only at the vertices $u$ and $v$. So, the 
% number of negative vertices in $\Sigma_\mu$ is even as it is so in $\Sigma^\prime_\mu$. Hence $\Sigma_\mu$  is consistent and 
% which in turn implies that $\Sigma$ is canonically consistent.
% 
% 
% % let a cycle has only one negative edge then,by canonically matking there is exactly two negative vertices and hence the cycle is 
% % canonically consistent.
% % 
% % So, let a cycle has exactly two negative edges then, by canonically matking the vertex corresponding to two negative incident 
% % edge is going to be positive and remaining two vertex will be negative and hence the cycle is canonically consistent.
% % let us suppose that the result holds for (n-1) negative edges.we prove the result fon n negative edges.
% % let us ulter a signed of positive edge in a cycle different from (n-1) negative edges.
% % If it is not incident with none of (n-1) negative edges then the result holds.
% % if it is incident with exactly one of (n-1) negative edge then,by canonically matking the signed of vertex coresponding to two 
% % negative incident edge is going to be positive and one vertex will be negative .Hence there is same no of vertices remains 
% % ,therefore the cycle is canonically consistent by assumption.
% % if it is incident with two negative edges then we get exactly two positive vertex and hence the cycle is canonically consistent.
% \end{proof}
% 
% \begin{proposition}\label{prop2}
%  A unicyclic signed graph is canonically consistent if and only if the number of negative 
% non-cyclic edges incident with the vertices of the cycle is even.
% \end{proposition}
% \begin{proof}
%  Let $\sum=(G,\sigma)$  be a unicyclic signed graph and $C$ be the cycle in  $\sum$.
%  First assume that  $\sum$ is canonically consistent. We need to prove that the number of negative 
% non-cyclic edges incident with the vertices of the cycle is even.
% Let $v_{i_1},v_{i_2},\cdots v_{i_p}$ be the vertices in $C$ on which negative non-cyclic edges are incident.
% Let $$X_k=\{e : \sigma(e) = -1  \text{ and } e \text{ is non-cyclic edge incident with } v_{i_k}\}$$ for $k=1,2,\cdots p.$
% Since $G$ is unicyclic, so $X_k$'s are pairwise disjoint and hence the number of negative non-cyclic edges incident with the 
% vertices of $C$ is $\sum\limits_{k=1}^p|X_k|$. If possible,  let $\sum\limits_{k=1}^p|X_k|$ be odd. Then 
% $\sum\limits_{k=1}^p|X_k|$ must have odd number of odd summands and this in turn implies that the number of negative vertices in 
% $C$ is odd, because in absence of all the non-cyclic edges incident with the vertices of $C$, the cycle is $C$ is consistent. 
% Hence, the graph $\sum=(G,\sigma)$ is inconsistent, a contradiction. Therefore,  $\sum\limits_{k=1}^p|X_k|$ must be even.
% 
% Conversely, let the number of negative 
% non-cyclic edges incident with the vertices of the cycle $C$ is even i.e., $\sum\limits_{k=1}^p|X_k|$ is even. Then, the number 
% of odd summands in $\sum\limits_{k=1}^p|X_k|$ must be even. Since, negative non-cyclic edges incident with a vertex $v$ of $C$ 
% will alter the sign of $v$ provided it is odd in number and every cycle is consistent, so the number of vertices with negative 
% marking in $C$ must be even in $G$. Hence, $C$ is consistent and so $G$ is consistent.
% 
% % 
% % We apply induction on the number of vertices $p$ of $C$ on which negative non-cyclic edges are incident.
% % 
% % Base Step: Let $p=0$. That is, no non-cyclic negative edge is incident with the vertices of $C$. In this case, sign of $C$ will 
% % depend on the signe of the edges involved in $C$ and so by Proposition \ref{prop1}, $C$ is consistent and hence $G$ is 
% % canonically consistent.
% 
% 
% % 
% % Let, $X=\{e : \sigma(e) = −1 \text{ or } + 1 \text{ and } e \in E(G)}$
% %  and $Y=\{e : \sigma(e) = −1 \text{ or } + 1 \text{ and } e \in E(C)}$
% % Let X − Y = e : e = −1and e is incident with the vertices of C ⊆ X
% % we claim that X − Y is even.
% % If mod X −Y is odd then,V = v : v = −1andv ∈ V (C) is odd and which makes the cycle
% % inconsisitent,which is a contradiction thta every cycle is canonically consistent.Hence,
% % mod X − Y is even.
% % Conversely, let mod X − Y is even then V = v : v = −1andv ∈ V (C) is even and hence,
% % is canonically consistent.
% 
% 
% % \red
% %  Let $\sum -\sigma=\{e:e=-1\},\sigma:E(C)\longrightarrow\{-1,+1\}$ be the negative non-cyclic edges incident with the vertices 
% % of 
% % $C$.
% % we claim that $\sigma:E(C)\longrightarrow\{-1,+1\}$ is even.
% % If $\sum -\sigma$ is odd,then  $\sigma:V(C)\longrightarrow\{-1,+1\}$ is odd,which is a contradiction that every cycle is 
% % canonically consistent.Hence  $\sum -\sigma$ is even.
% % conversely,
% % Let $\sum -\sigma$ is even.
% % Then $\sigma:V(C)\longrightarrow\{-1,+1\}$ is even and hence  $\sum$ is canonically consistent.
% \end{proof}
% \begin{theorem}
%  If all the cycles in a signed graph are disjoint, then the 
% signed graph is canonically consistent if and only if the number of non-cyclic negative edges incident with the vertices of each 
% cycles is even.
% \end{theorem}
% \begin{proof}
% The proof of this theorem follows from the Proposition \ref{prop2} by applying induction on the number of cycles in $G$.
% 
% 
% % We prove this result by applying induction on the number of cycles $c$ in $G$.
% % 
% % Base step: Let $c=1$. In this case $G$ is unicyclic and so the result is true by the Proposition \ref{prop2}.
% % 
% % Induction Step. As an induction hypothesis let us assume that the result is true for any  signed graph with fewer than $c$ cycles.
% % Let $e=uv$ by a non-cyclic edge appears on a path connecting two cycles $C_1,C_2$ in $G$. Then removal of $e$ will breack $G$ 
% % into two components, say, $G_1,G_2$.
% % We consider the following cases:
% % 
% % Case 1: $\sigma(e)=+1$. In this case, the number of negative non-cyclic edges incident with each cycle in $G_1$ or $G_2$ will 
% % remain same as that in $G$. By induction hypothesis, $G_i,~~i=1,2$ is consistent if and only if the number of negative non-cyclic 
% % edges incident with each cycle is even. Also, $G$ is consistent if and only if $G_1$ and $G_2$ are consistent. Hence, the result 
% % follows. 
% 
%  
% \end{proof}

\section{Conclusion}
In this dissertation we have shown that the sets $S_1,S_3,S_4$ are non empty. Further we have identified few classes of graphs 
that belongs to $S_1$ and $S_3$. But, we have not been able to get an example of a signed graph that belongs to $S_2$. We have a 
strong belief that $S_2$ is empty. 